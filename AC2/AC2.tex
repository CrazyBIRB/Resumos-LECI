\documentclass[a4paper, 11pt, onecolumn, twoside]{article}

\usepackage[utf8]{inputenc}
\usepackage{amsmath}
\usepackage{amssymb}
\usepackage{graphicx}
\usepackage[hidelinks]{hyperref}

\title{%
  Resumos de AC2 \\
  \large Teste Teorico 2}

\author{Tiago Almeida}
\date{\today}

\begin{document}

\maketitle

\tableofcontents

\section{Introdução}
Escrever um pequeno overview da matéria que sai para o teste teórico 2
e o que esperar encontrar neste documento

\section{A Interface \hyperref[sec:spi]{\textbf{\large{SPI}}}}
\hyperref[sec:spi]{\textbf{\large{SPI}}} é uma interface de \textbf{\large{alta velocidade e de curta distância}} (dezenas de cm) usada para comunicar com
diversos dispositivos diferentes, como por exemplo:
\begin{itemize}
    \item Sensores de diversos tipos: temperatura, pressão, etc.
    \item Cartões de memória (MMC / SD)
    \item Circuitos: memórias, ADCs, DACs, Displays LCD (e.g.\ telemóveis),
    comunicação entre corpo de máquinas fotográficas e as lentes, \ldots
    \item Comunicação entre microcontroladores
\end{itemize}

\subsection{Funcionamento}
Pontos chave sobre a arquitetura do \hyperref[sec:spi]{\textbf{\large{SPI}}}:\@
\begin{itemize}
    \item Comunicação full-duplex;
\end{itemize}

\subsection{Detalhes adicionais}
Algumas notas adicionais sobre \hyperref[sec:spi]{\textbf{\large{SPI}}} que podem ou não ser importantes:

\begin{itemize}
    \item Criado pela empresa Motorola
    \item Terceiro item da lista
    \begin{itemize}
        \item Subitem um
        \item Subitem dois
    \end{itemize}
    \item Quarto item da lista
\end{itemize}

\section{Conclusão}
Algumas conclusões e considerações que se deve ter após
ter acabado o estudo

\clearpage
\section{Glossário}\label{sec:glossary}

Aqui está a seção de glossário. Cada termo usado repetidamente no documento está listado aqui com sua definição.

\begin{itemize}
    \item\label{spi} \textbf{SPI}: Serial Peripheral Interface
    \item\label{halfduplex} \textbf{Half-Duplex}: comunicação nos dois sentidos, mas apenas um de
    cada vez (é usada uma só linha)
    \item\label{fullduplex} \textbf{Full-Duplex}: Comunicação simultânea nos dois sentidos (são
    usadas duas linhas)
\end{itemize}

\end{document}
